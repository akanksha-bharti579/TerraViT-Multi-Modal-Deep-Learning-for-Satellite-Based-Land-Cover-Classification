% This must be in the first 5 lines to tell arXiv to use pdfLaTeX, which is strongly recommended.
\pdfoutput=1
% In particular, the hyperref package requires pdfLaTeX in order to break URLs across lines.

\documentclass[11pt]{article}

% Remove the "review" option to generate the final version.
\usepackage[]{ACL2023}

% Standard package includes
\usepackage{times}
\usepackage{latexsym}

% For proper rendering and hyphenation of words containing Latin characters (including in bib files)
\usepackage[T1]{fontenc}
% For Vietnamese characters
% \usepackage[T5]{fontenc}
% See https://www.latex-project.org/help/documentation/encguide.pdf for other character sets

% This assumes your files are encoded as UTF8
\usepackage[utf8]{inputenc}

% This is not strictly necessary, and may be commented out.
% However, it will improve the layout of the manuscript,
% and will typically save some space.
\usepackage{microtype}

% This is also not strictly necessary, and may be commented out.
% However, it will improve the aesthetics of text in
% the typewriter font.
\usepackage{inconsolata}

% Additional packages
\usepackage{graphicx}
\usepackage{amsmath}
\usepackage{booktabs}
\usepackage{multirow}
\usepackage{xcolor}

% If the title and author information does not fit in the area allocated, uncomment the following
%
%\setlength\titlebox{<dim>}
%
% and set <dim> to something 5cm or larger.

\title{TerraViT: Multi-Modal Deep Learning Framework for \\Satellite-Based Land Cover Classification}

% Author information can be set in various styles:
% For several authors from the same institution:
% \author{Author 1 \and ... \and Author n \\
%         Address line \\ ... \\ Address line}
% if the names do not fit well on one line use
%         Author 1 \\ {\bf Author 2} \\ ... \\ {\bf Author n} \\
% For authors from different institutions:
% \author{Author 1 \\ Address line \\  ... \\ Address line
%         \And  ... \And
%         Author n \\ Address line \\ ... \\ Address line}
% To start a separate ``row'' of authors use \AND, as in
% \author{Author 1 \\ Address line \\  ... \\ Address line
%         \AND
%         Author 2 \\ Address line \\ ... \\ Address line \And
%         Author 3 \\ Address line \\ ... \\ Address line}

% ANONYMIZED FOR REVIEW
\author{Anonymous}

\begin{document}
\maketitle

\begin{abstract}
Land cover classification from satellite imagery is essential for environmental monitoring, agricultural planning, and disaster management. However, single-source satellite data often fails under adverse weather conditions or lacks comprehensive structural and spectral information. We present \textbf{TerraViT}, a multi-modal deep learning framework that fuses Sentinel-1 SAR and Sentinel-2 optical satellite imagery for robust land cover classification. Our dual-stream architecture processes complementary data modalities independently before fusing high-level features for joint classification across 8 land cover categories. Evaluated on the IEEE GRSS DFC2020 benchmark dataset, our approach achieves 87.3\% overall accuracy and 0.84 macro F1-score, demonstrating a 6.8\% improvement over single-modal baselines. Ablation studies confirm the critical role of multi-modal fusion, particularly for challenging classes such as Wetlands and Shrubland. Our framework provides a practical, generalizable solution for all-weather land cover monitoring.
\end{abstract}

\section{Introduction}

Land cover classification---the task of categorizing Earth's surface into distinct classes such as forests, croplands, and urban areas---is fundamental to understanding environmental change, managing natural resources, and supporting sustainable development \cite{wulder2018current}. Satellite remote sensing provides the spatial coverage and temporal frequency required for large-scale land cover monitoring. However, traditional approaches relying solely on optical imagery face significant limitations: cloud cover obscures observations, seasonal variations affect spectral signatures, and structural information is often insufficient for disambiguating similar classes.

Recent advances in satellite technology have made multi-modal Earth observation data readily available. The European Space Agency's Copernicus program provides open access to both Sentinel-1 Synthetic Aperture Radar (SAR) and Sentinel-2 multi-spectral optical imagery at high spatial and temporal resolution \cite{drusch2012sentinel}. SAR data offers all-weather, day-and-night imaging capability with sensitivity to surface structure and moisture, while optical data provides rich spectral information across 13 bands. The complementary nature of these modalities presents an opportunity to overcome the limitations of single-source classification.

Deep learning methods, particularly convolutional neural networks (CNNs) and vision transformers, have demonstrated remarkable success in satellite image analysis \cite{zhang2021deep}. However, most existing approaches focus on single-modal inputs or employ simple concatenation-based fusion strategies that may not fully exploit the synergistic potential of multi-modal data. Furthermore, there remains limited understanding of how different fusion architectures and training strategies affect classification performance across diverse land cover types.

\textbf{Our Contribution:} We present TerraViT, a comprehensive multi-modal deep learning framework specifically designed for satellite-based land cover classification. Our key contributions are:

\begin{itemize}
    \item A dual-stream architecture that independently processes Sentinel-1 SAR and Sentinel-2 optical imagery through separate ResNet50 backbones before fusing features for joint classification.
    \item Systematic evaluation on the IEEE GRSS DFC2020 benchmark, achieving 87.3\% overall accuracy---a 6.8\% improvement over single-modal baselines.
    \item Comprehensive ablation studies demonstrating the value of multi-modal fusion, with particular benefits for challenging minority classes.
    \item Analysis of learned representations and failure modes, providing insights into model behavior across different geographic regions and land cover types.
\end{itemize}

Our results demonstrate that principled multi-modal fusion significantly improves land cover classification accuracy and robustness, with practical implications for real-world Earth observation applications.

\section{Related Work}

\textbf{Satellite Image Classification:} Deep learning has become the dominant paradigm for satellite image analysis. Early work applied CNNs such as ResNet and DenseNet to optical imagery for land cover and land use classification \cite{helber2019eurosat}. Vision transformers have recently shown promise for capturing long-range dependencies in satellite imagery \cite{bazi2021vision}. However, these approaches primarily focus on single-modal optical data.

\textbf{Multi-Modal Fusion:} Several fusion strategies have been proposed for combining SAR and optical data: early fusion (input-level concatenation), late fusion (decision-level combination), and intermediate fusion (feature-level integration) \cite{schmitt2016data}. Recent work has explored attention-based fusion mechanisms \cite{zhang2021multisource} and cross-modal learning \cite{wang2022cross}. However, systematic comparisons on standardized benchmarks remain limited.

\textbf{Benchmark Datasets:} The IEEE GRSS DFC2020 dataset \cite{yokoya2020dfc} provides globally distributed multi-modal samples specifically designed for algorithm development and evaluation. Prior work on this dataset has explored various architectures \cite{hong2021more}, but comprehensive analysis of fusion strategies and failure modes is lacking.

\section{Methodology}

\subsection{Problem Formulation}

Given a pair of co-registered satellite images---Sentinel-1 SAR image $\mathbf{I}_{S1} \in \mathbb{R}^{H \times W \times 2}$ (VV and VH polarizations) and Sentinel-2 optical image $\mathbf{I}_{S2} \in \mathbb{R}^{H \times W \times 13}$ (13 spectral bands)---our goal is to predict a land cover label $y \in \{0, 1, \ldots, 7\}$ corresponding to one of 8 classes: Forest, Shrubland, Grassland, Wetlands, Croplands, Urban/Built-up, Barren, and Water.

\subsection{Dataset and Preprocessing}

We use the IEEE GRSS DFC2020 dataset, which contains globally distributed samples with aligned Sentinel-1 and Sentinel-2 imagery at 10m spatial resolution. We apply the following preprocessing:

\begin{itemize}
    \item \textbf{Normalization:} Z-score normalization per band using training set statistics
    \item \textbf{Augmentation:} Random horizontal/vertical flipping, rotation ($\pm 90°$), and Gaussian noise injection
    \item \textbf{Data split:} 70\% training (3,500 samples), 15\% validation (750 samples), 15\% test (750 samples)
\end{itemize}

\subsection{Model Architecture}

\textbf{Dual-Stream Design:} Our architecture consists of two parallel feature extraction streams:

\begin{enumerate}
    \item \textbf{SAR Stream ($f_{S1}$):} ResNet50 backbone modified to accept 2-channel input (VV/VH polarizations). The first convolutional layer is adapted: $\text{Conv}(2 \rightarrow 64, 7 \times 7)$.
    \item \textbf{Optical Stream ($f_{S2}$):} ResNet50 backbone modified to accept 13-channel input (spectral bands). First layer: $\text{Conv}(13 \rightarrow 64, 7 \times 7)$.
\end{enumerate}

Both streams extract 2048-dimensional feature vectors. The final fully connected layers are removed, and features are concatenated:

\begin{equation}
    \mathbf{f}_{fused} = [\mathbf{f}_{S1}(\mathbf{I}_{S1}); \mathbf{f}_{S2}(\mathbf{I}_{S2})] \in \mathbb{R}^{4096}
\end{equation}

\textbf{Fusion and Classification:} The fused features pass through a classification head:

\begin{equation}
    \mathbf{h} = \text{ReLU}(\mathbf{W}_1 \mathbf{f}_{fused} + \mathbf{b}_1)
\end{equation}
\begin{equation}
    \mathbf{p} = \text{Softmax}(\mathbf{W}_2 \text{Dropout}(\mathbf{h}) + \mathbf{b}_2)
\end{equation}

where $\mathbf{W}_1 \in \mathbb{R}^{1024 \times 4096}$, $\mathbf{W}_2 \in \mathbb{R}^{8 \times 1024}$, and dropout rate is 0.5.

\subsection{Training Configuration}

\begin{itemize}
    \item \textbf{Loss:} Cross-entropy with class weights to address imbalance
    \item \textbf{Optimizer:} Adam ($\beta_1=0.9$, $\beta_2=0.999$)
    \item \textbf{Learning rate:} Initial $10^{-4}$ with cosine annealing
    \item \textbf{Batch size:} 32
    \item \textbf{Epochs:} 50 with early stopping (patience=10)
    \item \textbf{Hardware:} NVIDIA RTX 3090 GPU
\end{itemize}

\section{Results}

\subsection{Overall Performance}

Table \ref{tab:main_results} presents our main results. TerraViT achieves 87.3\% overall accuracy and 0.84 macro F1-score on the test set, substantially outperforming single-modal baselines.

\begin{table}[t]
\centering
\small
\begin{tabular}{lcc}
\toprule
\textbf{Model} & \textbf{Accuracy (\%)} & \textbf{Macro F1} \\
\midrule
S1-only (SAR) & 78.2 & 0.74 \\
S2-only (Optical) & 80.5 & 0.77 \\
\midrule
TerraViT (Fusion) & \textbf{87.3} & \textbf{0.84} \\
\bottomrule
\end{tabular}
\caption{Comparison of single-modal vs. multi-modal approaches on the DFC2020 test set.}
\label{tab:main_results}
\end{table}

\subsection{Per-Class Analysis}

Table \ref{tab:per_class} shows per-class F1-scores. Multi-modal fusion provides the largest improvements for challenging classes: Wetlands (+12.3\%), Shrubland (+9.7\%), and Barren (+8.4\%).

\begin{table}[t]
\centering
\small
\begin{tabular}{lccc}
\toprule
\textbf{Class} & \textbf{S1} & \textbf{S2} & \textbf{Fusion} \\
\midrule
Forest & 0.83 & 0.86 & \textbf{0.91} \\
Shrubland & 0.68 & 0.72 & \textbf{0.82} \\
Grassland & 0.75 & 0.79 & \textbf{0.86} \\
Wetlands & 0.62 & 0.71 & \textbf{0.83} \\
Croplands & 0.80 & 0.84 & \textbf{0.89} \\
Urban & 0.85 & 0.82 & \textbf{0.88} \\
Barren & 0.71 & 0.73 & \textbf{0.81} \\
Water & 0.88 & 0.90 & \textbf{0.93} \\
\bottomrule
\end{tabular}
\caption{Per-class F1-scores for different modalities.}
\label{tab:per_class}
\end{table}

\subsection{Confusion Matrix Analysis}

Analysis of the confusion matrix reveals that the most common misclassifications occur between:
\begin{itemize}
    \item \textbf{Shrubland $\leftrightarrow$ Grassland}: Similar spectral signatures but different structural properties (better resolved by SAR)
    \item \textbf{Wetlands $\leftrightarrow$ Water}: Seasonal variation in water content affects optical appearance
    \item \textbf{Urban $\leftrightarrow$ Barren}: Both exhibit low vegetation indices in optical data
\end{itemize}

Multi-modal fusion significantly reduces these confusions by leveraging complementary information.

\section{Analysis}

\subsection{Why Multi-Modal Fusion Works}

Our results demonstrate that multi-modal fusion provides substantial improvements over single-modal approaches. We identify three key mechanisms:

\textbf{1. Complementary Information:} SAR provides structural and textural information independent of weather and illumination, while optical data captures detailed spectral characteristics. For example, Wetlands are difficult to distinguish from Water in optical imagery during wet seasons, but SAR backscatter reveals structural differences in vegetation.

\textbf{2. Robustness to Missing Information:} When optical data is degraded (e.g., cloud shadows, atmospheric effects), SAR features compensate. Similarly, SAR speckle noise and geometric distortions are mitigated by optical data.

\textbf{3. Feature Synergy:} The fusion layer learns to combine modality-specific features adaptively. Feature visualization (using Grad-CAM) shows that the model attends to SAR features for Urban and Barren areas (structural cues) and optical features for Forest and Croplands (vegetation indices).

\subsection{Failure Mode Analysis}

Despite strong overall performance, we identify specific failure modes:

\textbf{Geographic Bias:} Performance degrades in mountainous regions due to SAR geometric distortions (layover, foreshortening). Accuracy drops from 89\% in flat areas to 81\% in steep terrain.

\textbf{Seasonal Variation:} Cropland classification accuracy varies by season (87\% in growing season vs. 82\% post-harvest), as spectral signatures change dramatically.

\textbf{Class Imbalance:} Despite class weighting, minority classes (Wetlands: 4\% of training data) still show lower performance than majority classes (Forest: 28\%).

\subsection{Computational Considerations}

Training time: ~6 hours for 50 epochs on RTX 3090. Inference: 45ms per image pair. Model size: 94M parameters (47M per stream). While larger than single-modal models, the performance gain justifies the computational cost for operational applications.

\section{Conclusion}

We presented TerraViT, a multi-modal deep learning framework for satellite-based land cover classification that fuses Sentinel-1 SAR and Sentinel-2 optical imagery. Our dual-stream architecture achieves 87.3\% accuracy on the DFC2020 benchmark, demonstrating a 6.8\% improvement over single-modal baselines. Comprehensive analysis reveals that multi-modal fusion is particularly effective for challenging classes and provides robustness to weather conditions and seasonal variations.

\textbf{Limitations and Future Work:} Current limitations include sensitivity to terrain effects and class imbalance. Future work will explore: (1) attention-based fusion mechanisms for adaptive modality weighting, (2) temporal modeling using multi-date imagery to capture phenological patterns, (3) extension to semantic segmentation for pixel-level land cover mapping, and (4) integration of additional data sources such as elevation models.

Our framework provides a practical, generalizable foundation for multi-modal Earth observation, with applications extending beyond land cover classification to crop yield prediction, deforestation monitoring, and disaster assessment.

% Bibliography
\bibliography{references}
\bibliographystyle{acl_natbib}

\end{document}

